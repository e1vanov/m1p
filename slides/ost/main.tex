\documentclass{beamer}

\usepackage{wasysym}
\usepackage[center]{caption}

\documentclass[twocolumn]{article}

\usepackage[T2A]{fontenc}
\usepackage[utf8]{inputenc}
\usepackage{amssymb,amsmath,mathrsfs,amsthm}
\usepackage[russian]{babel}
\usepackage{graphicx}
\usepackage{indentfirst}
\usepackage{lipsum}
\usepackage{caption}
\captionsetup{justification=centering}

\usepackage{multicol}
\setlength{\columnsep}{1cm}

\usepackage[style=nature]{biblatex}
\addbibresource{sources.bib}
%\addbibresource{Paper_Ivanov.bib}

\usepackage{geometry}
\geometry{top=25mm}
\geometry{bottom=30mm}
\geometry{left=20mm}
\geometry{right=20mm}

\renewcommand{\baselinestretch}{1.1}
\renewcommand{\baselinestretch}{1.1}

\usepackage{authblk}


\title[OST. Фонтанные коды]{Использование идей фонтанного кодирования 
при передаче малого числа битовых пакетов}
\author[Иванов Е.Р.]{Иванов Егор Романович, 317 группа \\ 
Научный руководитель: к.ф.-м.н. Гуров Сергей Исаевич}
\date{\today}
\institute{ММП ВМК МГУ}

\titlegraphic{\includegraphics[scale=0.12]{img/msu.eps}}
\usetheme{Madrid}

\begin{document}

\begin{frame}{One-slide Talk}

    \begin{tabular}{cl}  
         \begin{tabular}{c}
           \includegraphics[width=0.35\linewidth]{img/lt_process.jpg}
           \end{tabular}
           & \begin{tabular}{l}
             \parbox{0.53\linewidth}{%  change the parbox width as appropiate
             Задача передачи битовых пакетов по каналу связи:
             \begin{itemize}
                 \item код \textit{систематический}
                 \item модель канала -- \textit{<<очень хороший>>} BEC
                 \item при одних и тех же параметрах высылается \textit{малое} число пакетов
             \end{itemize}
             \textbf{Требуется:} используя идеи фонтанного кодирования, сделать канал \textit{идеальным}.
             }
    \end{tabular}  \\
    \end{tabular}

\end{frame}

\end{document}
